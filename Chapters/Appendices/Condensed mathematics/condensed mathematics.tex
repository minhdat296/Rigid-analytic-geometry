\section{The condensed formalism} \label{section: condensed_mathematics}
    \subsection{Introduction to the condensed formalism}
        \subsubsection{Condensed sets}
            \begin{definition}[Condensation] \label{def: condensation}
                Let $\kappa$ be a fixed strong limit cardinal and let $\C$ be a hypercomplete $\infty$-category with enough $\kappa$-small limits and enough $\kappa$-small filtered colimits. We then define so-called \textbf{condensed objects} of $\C$ to be $\C$-valued sheaves over the $\kappa$-small pro-\'etale site of a point (i.e. the pro-\'etale site of the spectrum of a field). 
                
                Clearly condensed objects of a given $\infty$-category $\C$ satisfying the above conditions form a category. We shall denote it by $\C^{\cond}$.
            \end{definition}
            \begin{remark}[The pro-\'etale topology for schemes]
                For now, we refer the reader to \cite[Definition 4.1.1 and Remark 4.1.3]{bhatt_scholze_2014_pro_etale} for the definition of the pro-\'etale topology for schemes. In particular, recall that the $\kappa$-small pro-\'etale site of a point is equivalent to the site $\Pro_{\kappa}(\Fin\Sets)$ of $\kappa$-small profinite sets\footnote{Which we viewed as totally disconnected $\kappa$-small compact Hausdorff spaces.}, whose coverage is generated by jointly surjective finite families. 
            \end{remark}
            \begin{example}
                \noindent
                \begin{itemize}
                    \item \textbf{(\textit{Small} condensed sets):} The pro-\'etale topos $\Sh(*_{\kappa\-\proet})$ over a point is, by definition, the category of sheaves of sets on the pro-\'etale site of a point. Therefore, this topos is the category of \textbf{$\kappa$-small condensed sets}. Whenenver we wish to put emphasis on the fact that $\Sh(*_{\kappa\-\proet})$ is actually the category of $\kappa$-small condensed sets, we will write $\kappa\-\Sets^{\cond}$ instead.
                    \item \textbf{(Condensed abelian groups and modules):} If $R$ is a condensed commutative ring, then the category $(R\mod^{\cond})^{\heart}$ of condensed $R$-modules is a \href{https://ncatlab.org/nlab/show/Grothendieck+category}{\underline{Grothendieck category}}; on the other hand, topological abelian groups even fails to form an abelian category, and as we well know: no abelian categories means no homological algebra. This is a \textit{biggy}, so we shall bestow upon it the dignity of theorem-hood.
                \end{itemize}
            \end{example}
            \begin{remark}[Condensedness and profiniteness] \label{remark: condensedness_and_profiniteness}
                Fix a strong limit cardinal $\kappa$. The $\kappa$-small pro-\'etale site of a point is equivalent to the site $\kappa\-\Pro\Fin\Sets_{\surj} \cong \kappa\-\Pro(\Fin\Sets_{\surj})$ whose underlying category is that of $\kappa$-small profinite sets, and whose coverage is generated by jointly surjective finite families of functions (hence the subscript). One can show this using the fact that the $\kappa$-small pro-\'etale site of a point is the same as the $\kappa$-small pro-\'etale site of the spectrum of a field, which is nothing but the $\kappa$-small pro-completion of the (small) finite \'etale site of a field (and recall that this site is equivalent to the category of finite sets equipped with the coverage given by jointly surjective finite families). 
                
                This identification of the category of $\kappa$-small profinite sets with the $\kappa$-small pro-\'etale site of a point allows us to re-interpret the sheaf condition defining condensed sets in the following more practical manner: a $\kappa$-small condensed set is a presheaf:
                    $$X: \kappa\-\Pro\Fin\Sets_{\surj}^{\op} \to \Sets$$
                such that $X(\varnothing) \cong \{*\}$, $X(S \sqcup S') \cong X(S) \x (S')$ for all $S, S' \in \Ob(\kappa\-\Pro\Fin\Sets)$, and that for any surjection $\pi: S' \to S$ between $\kappa$-small profinite sets $S, S'$, the following diagram commutes in $\Sets$:
            \end{remark}
            
            \begin{remark}[Set-theoretic issues and technicalities] \label{remark: condensed_sets_set_theoretic_issues}
                It might seem as though the fixture of a strong limit cardinal $\kappa$ is an unnecessary gimmick and that the issues that one might run into when removing this cardinal bound are purely philosophical. However, because the unbounded pro-\'etale site $*_{\proet}$ (or for that matter, the category of all profinite sets) is large, it is not guaranteed that a sheafification functor ${}^{\sh}(-): \Psh(*_{\proet}) \to \Sh(*_{\proet})$ would exist as the left-adjoint of the natural inclusion $\Psh(*_{\proet}) \subset \Sh(*_{\proet})$. Since many of the algebraic constructions in the condensed formalism (e.g. condensed rings, condensed modules, etc.) benefit from being objects internal to some hypothetical small topos\footnote{All topoi shall be sheaf topoi, i.e. Grothendieck topoi (cf. \cite[Expos\'e IV]{sga4}).} (as pro-\'etale sheafification is a somewhat tricky procedure), it will be worth our while to impose a cardinal bound $\kappa$ onto the underlying pro-\'etale site $*_{\proet}$ in order to obtain a $\kappa$-small topos $\kappa\-\Sets^{\cond}$ of $\kappa$-small condensed sets.
                
                In order to impose such a size bound on $*_{\proet}$ (or rather, the large category of all profinite sets), let us first of all fix a(n) (uncountable) strong limit cardinal $\kappa$ (i.e. $\kappa$ is uncountable and for all cardinals $\lambda < \kappa$, we also have $2^{\lambda} < \kappa$), which can be constructed as follows: define $\beta_0 := \aleph_0$, and then for all ordinals $\alpha$, define $\beth_{\alpha}$ inductively via $\beth_{\alpha^+} := 2^{\beth_{\alpha}}$, which would then ensure that for any fixed ordinals $\alpha_0$, the cardinal $\kappa := \beth_{\alpha_0}$ is a strong limit cardinal. $*_{\kappa\-\proet}$ can then be simply taken as the category of all $\kappa$-small profinite sets (cf. remark \ref{remark: condensedness_and_profiniteness}). 
                
                Now, if $\kappa < \kappa'$ are strong limit cardinals then there is an obvious inclusion $\kappa\-\Pro\Fin\Sets_{\surj} \subset \kappa'\-\Pro\Fin\Sets_{\surj}$ of the category of $\kappa$-small profinite sets into that of $\kappa'$-small profinite sets. This functor, in turn, induces a forgetful functor\footnote{Due to the fact that over any site $\C$, there is a fibration $\Sh \to \C^{\op}$ given by $(f: Y \to X) \mapsto (f^*: \Sh({\C_{/X}}) \to \Sh(\C_{/Y}))$.}:
                    $$\oblv_{\kappa < \kappa'}: \kappa'\-\Sets^{\cond} \to \kappa\-\Sets^{\cond}$$
                which by general topos theory (cf. \cite[Expos\'e IV]{sga4}) admits a left-exact left-adjoint:
                    $$L_{\kappa < \kappa'}: \kappa\-\Sets^{\cond} \to \kappa'\-\Sets^{\cond}$$
                What can be shown is that for every pair $\kappa < \kappa'$ of strong limit cardinals, this left-adjoint is actually fully faithful (cf. proposition \ref{prop: increasing_the_cardinal_bound_on_condensed_sets}), thus allowing us to define the large category of all condensed sets (regardless of their cardinal bounds) as the filtered colimit along these fully faithful left-adjoints:
                    $$\Sets^{\cond} \cong \underset{ \text{Strong limit cardinals $\kappa$} }{\colim} \kappa\-\Sets^{\cond}$$
            \end{remark}
            \begin{definition}[Compacta] \label{def: compacta}
                Let $\kappa$ be a strong limit cardinal. The category of ($\kappa$-small) \textbf{compacta} (singular: \textbf{compactum}), which we denote by $\kappa\-\Comp$, is then taken to be that of ($\kappa$-small) compact Hausdorff topological spaces and continuous functions between them. 
            \end{definition}
            \begin{convention}[The site of compacta] \label{conv: site_of_compacta}
                From now on, $\Comp_{\surj}$ (respectively, $\kappa\-\Comp_{\surj}$) shall denote the site of ($\kappa$-small) compacta on which the coverage is given by jointly surjective finite families.
            \end{convention}
            \begin{lemma}[Condensed sets as sheaves on compacta] \label{lemma: condensed_sets_as_sheaves_on_compacta}
                For any strong limit cardinal $\kappa$, there is a canonical equivalence of topoi $\kappa\-\Sets^{\cond} \cong \Sh(\kappa\-\Comp_{\surj})$.
            \end{lemma}
                \begin{proof}
                    
                \end{proof}
            \begin{proposition}[Increasing the cardinal bound on condensed sets] \label{prop: increasing_the_cardinal_bound_on_condensed_sets}
                
            \end{proposition}
                \begin{proof}
                    
                \end{proof}
                
            Let us now take a slight break from attempting to establish the formal properties of condensed sets to take a make a comparison between them and traditional topological spaces: after all, the main reason behind the introduction of the notion of condensed sets is to provide an ambient topos wherein constructions such as condensed groups, rings, modules, etc. might effectively replace usual topological groups, rings, modules, etc., seeing how the latter enjoy less-than-favourable formal properties (e.g. the category of topological abelian groups is not even abelian).
            \begin{remark}[Topological spaces vs. condensed sets] \label{remark: topological_spaces_and_condensed_sets}
                Fix a strong limit cardinal $\kappa$. To any $\kappa$-small topological space $T$, there is an associated $\kappa$-small condensed set $T^{\kappa\-\cond}$, which is the sheafification of the presheaf\footnote{The readers can check for themselves that indeed, the presheaf $S \mapsto \Maps(S, T)$ is a sheaf.} on $\kappa\-\Pro\Fin\Sets_{\surj}$ given by $S \mapsto \Maps(S, T)$. Any topological group, ring, module, etc. is as such a condensed group, ring, module, etc. in a natural manner. 
                
                In addition, we note that the association:
                    $$(-)^{\kappa\-\cond}: \kappa\-\Top \to \kappa\-\Sets^{\cond}$$
                    $$T \mapsto T^{\kappa\-\cond}$$
                of a $\kappa$-small topological space $T$ to its associated $\kappa$-small condensed set $T^{\kappa\-\cond}$ in the aforementioned manner is functorial (this is a Yoneda-type argument). For this reason, we have grounds for making definition \ref{def: condensation_of_topological_spaces}. 
            \end{remark}
            \begin{definition}[Condensation] \label{def: condensation_of_topological_spaces}
                For a fixed a strong limit cardinal $\kappa$, there is a \textbf{condensation} functor:
                    $$(-)^{\kappa\-\cond}: \kappa\-\Top \to \kappa\-\Sets^{\cond}$$
                    $$T \mapsto T^{\kappa\-\cond}$$
                constructed as in remark \ref{remark: topological_spaces_and_condensed_sets}.
            \end{definition}
            \begin{definition}[Compactly generated topological space] \label{def: compactly_generated_topological_spaces}
                Let $\kappa$ be a strong limit cardinal. The category $\<\Comp\>$ of \textbf{compactly generated topological spaces} (respectively, the category $\kappa\-\<\Comp\>$ of \textbf{$\kappa$-small compactly generated topological spaces}) and continuous functions between them is the ($\kappa$-small) free cocompletion of the category of ($\kappa$-small) compacta. 
            \end{definition}
            \begin{remark}
                Because the category of compacta is large (it contains the large category of all profinite sets as a full subcategory), its free cocompletion is not equivalent to the presheaf category $\Psh(\Comp)$. Of course, the category of compactly generated topological spaces is also large. 
            \end{remark}
            \begin{lemma}[The Stone-\v{C}ech Compactification] \label{lemma: the_stone_cech_compactification}
                The natural fully faithful embedding $\Comp \subset \Top$ admits a left-adjoint $\beta: \Top \to \Comp$, commonly known as the \textbf{Stone-\v{C}ech Compactification}.
            \end{lemma}
                \begin{proof}
                    
                \end{proof}
            \begin{definition}[Extremally disconnected spaces] \label{def: extremally_disconnected_spaces}
                For any fixed strong limit cardinal $\kappa$, the category $\Extr$ (respectively, $\kappa\-\Extr$) of ($\kappa$-small) \textbf{extremally disconnected spaces} is the full subcategory of $\Comp$ (respectively, $\kappa\-\Comp$) spanned by projective objects (i.e. if $T$ is extremally disconnected then any surjective map $S \to T$ from a compactum $S$ splits). 
            \end{definition}
            \begin{example}[Stone-\v{C}ech Compactifications of discrete spaces] \label{example: stone_cech_compactifications_of_discrete_spaces}
                The Stone-\v{C}ech compactification $\beta(T)$ of any discrete topological space $T$ is a profinite set (hence totally disconnected, hence extremally disconnected). In fact, any extremally disconnected space exists as a retract of some Stone-\v{C}ech Compactification.
            \end{example}
            \begin{convention}[The site of extremally disconnected spaces] \label{conv: the_site_of_extremally_disconnected_spaces}
                Henceforth, we shall write $\Extr_{\surj}$ (respectively, $\kappa\-\Extr_{\surj}$) for the site of ($\kappa$-small) extremally disconnected spaces on which the topology is given by jointly surjective finite families.
            \end{convention}
            \begin{proposition}[Condensed sets as sheaves on extremally disconnected spaces] \label{prop: condensed_sets_are_sheaves_on_extremally_disconnected_spaces}
                For any strong limit cardinal $\kappa$, there is a natural equivalence of topoi $\kappa\-\Sets^{\cond} \cong \Sh(\kappa\-\Extr_{\surj})$. Furthermore, there is a natural equivalence of large topoi $\Sets^{\cond} \cong \Sh(\Extr_{\surj})$.
            \end{proposition}
                \begin{proof}
                    
                \end{proof}
            \begin{remark}
                There is, of course, a natural fully faithful embedding $\<\Comp\> \subset \Top$ of the category of compactly generated topological spaces into that of all topological spaces. Interestingly, this embedding admits a right-adjoint (i.e. $\<\Comp\>$ is a coreflective subcategory of $\Top$), given by $T \mapsto \left(\coprod_{S \in \Ob(\Comp_{/T})} S \to T\right)$: what we can show is that for any topological space $T$, the canonically induced map $\coprod_{S \in \Ob(\Comp_{/T})} S \to T$ is surjective and as such one can endow $T$ with the quotient topology inherited from $\coprod_{S \in \Ob(\Comp_{/T})} S$, and since each $S$ is a compactum, doing so ensures that $T$ is actually a compactly generated topological space in the sense of definition \ref{def: compactly_generated_topological_spaces}. 
            \end{remark}
            \begin{theorem}[Condensation is (fully) faithful] \label{theorem: condensation_of_topological_spaces_is_fully_faithful}
                For any fixed strong limit cardinal $\kappa$, the functor $(-)^{\kappa\-\cond}: \kappa\-\Top \to \kappa\-\Sets^{\cond}$ as in definition \ref{def: condensation} is faithful, and the restriction $(-)^{\kappa\-\cond}|_{\kappa\-\<\Comp\>}: \kappa\-\<\Comp\> \to \kappa\-\Sets^{\cond}$ down onto $\kappa$-small compactly generated topological spaces is furthermore full (hence fully faithful); in fact, the condensation $T^{\kappa\-\cond}$ of any $\kappa$-small compactum $T$ is a $\kappa$-small qcqs\footnote{A condensed set is qcqs if and only if it is so when viewed as a sheaf of sets on $(\Spec k)_{\proet}$ for some field $k$.} condensed set.
            \end{theorem}
                \begin{proof}
                    
                \end{proof}
            \begin{theorem}[Condensation is a right-adjoint] \label{theorem: condensation_of_topological_spaces_is_a_right_adjoint}
                For any fixed strong limit cardinal $\kappa$, the functor $(-)^{\kappa\-\cond}: \kappa\-\Top \to \kappa\-\Sets^{\cond}$ as in definition \ref{def: condensation} admits a left-adjoint (which we shall call \textbf{decondensation}):
                    $$(-)^{\kappa\-\decond}: \kappa\-\Sets^{\cond} \to \kappa\-\Top$$
                    $$X \mapsto X(\{*\})$$
                wherein the topology on the space $X^{\kappa\-\decond} := X(\{*\})$ (note that indeed $\{*\} \in \Ob(\kappa\-\Pro\Fin\Sets)$) is understood to be equipped with the quotient topology coming from $\coprod_{S \in \Ob(\kappa\-\Pro\Fin\Sets_{/X^{\kappa\-\decond}})} S \to X^{\kappa\-\decond}$.
            \end{theorem}
                \begin{proof}
                    
                \end{proof}
            \begin{corollary}
                For any strong limit cardinal $\kappa$, the pair $(-)^{\kappa\-\decond} \ladjoint (-)^{\kappa\-\cond}$ as in theorems \ref{theorem: condensation_of_topological_spaces_is_fully_faithful} and \ref{theorem: condensation_of_topological_spaces_is_a_right_adjoint} is an adjoint equivalence between the category of $\kappa$-small compacta and that of $\kappa$-small qcqs condensed sets.  
            \end{corollary}
            
        \subsubsection{Condensed abelian sheaf cohomology}
            \begin{convention}[Regarding derived categories of modules] \label{conv: derived_categories_of_condensed_modules}
                Let us note that elsewhere, for $R$ any ring, the notation $R\mod$ shall be used for denoting the derived category of $R$-modules (and $R\mod^{\leq 0}, R\mod^{\geq 0}, R\mod^-, R\mod^+, R\mod^b$ shall mean the usual truncated/bounded subcategories), whereas $R\mod^{\heart}$ shall mean the \textit{underived} category of $R$-modules\footnote{The notation is justified as the underived category of $R$-modules is the abelian category that is the heart of the canonical t-structure on the derived category $R\mod$.}. 
                
                In accordance with this convention for ordinary modules, if $R$ is a condensed ring then the \textit{underived} category of (condensed) $R$-modules shall be denoted by $(R\mod^{\cond})^{\heart}$, whereas $R\mod^{\cond}$ (respecitvely, $(R\mod^{\cond})^{\leq 0}, (R\mod^{\cond})^{\geq 0}, (R\mod^{\cond})^-, (R\mod^{\cond})^+, (R\mod^{\cond})^b$) shall mean the \textit{derived} category of (condensed) $R$-modules (respectively, its truncated/bounded subcategories).
            \end{convention}
        
    \subsection{Symmetric monoidal structures; solidity}
        \begin{convention}[Condensed local systems]
            From now on, if $L \in \Sets$ is a set then the corresponding condensed local system (i.e. pro-\'etale local system over a point) shall be suggestively denoted by $L^{\cond}$.
        \end{convention}
    
        First of all, let us clarify that it is not that tensor products of condensed modules do not exist. However, such tensor products will usually end up being topologically pathological or just outright nonsensical. Take for instance, the local system $\Z_p^{\cond} \in \Sets^{\cond}$. Its tensor product with other non-archimedean local systems can be easily topologised via formal completion, but if we were to consider say, $\Z_p^{\cond} \tensor \R^{\cond}$ or $\Z_p^{\cond} \tensor \Z_{\ell}^{\cond}$ (where $\ell \not = p$ is another prime), then it is not very clear what the corresponding topological completion should be.  