\section{Cotangent complexes and smoothness}
    \subsection{Cotangent complexes}
        \subsubsection{K\"ahler differentials} \label{subsubsection: kahler_differentials}
            \begin{definition}[K\"ahler differentials] \label{def: kahler_differentials}
                Let $R$ be a commutative ring and let $\varphi: R \to S$ be a commutative $R$-algebra. An \textbf{$R$-derivation} (or simply \textbf{derivation} when the base ring $R$ is understood from context) from $S$ into an $S$-module $N$ is an $R$-linear map:
                    $$d: S \to N$$
                such that $d(\varphi(a)) = 0$ for all $a \in R$ and such that $d(fg) = fd(g) + d(f)gs$ for all $f, g \in S$.
            \end{definition}
            \begin{lemma}[Modules of derivations] \label{lemma: modules_of_derivations}
                Let $R$ be a commutative ring and let $\varphi: R \to S$ be a commutative $R$-algebra. The set $\Der_R(S, N)$ of all $R$-derivations from $S$ into a fixed $S$-module $N$ thus carries a natural $S$-module structure. Furthermore, the assignment:
                    $$\Der_R(S, -): S\mod \to S\mod$$
                is a covariant functor which is represented by an $S$-module $\Omega^1_{S/R}$, generated by the symbols $d(f)$ (for all $f \in S$) and determined by the relations $d(f) + d(g) = d(f + g), fd(g) + d(f)g = d(fg), d(\varphi(a)) = 0$ (for all $f, g \in S$ and all $a \in R$).
            \end{lemma}
                \begin{proof}
                    That $\Der_R(S, N)$ is trivial, so let us focus on showing that there is a well-defined functor $\Der_R(S, -): S\mod \to S\mod$ that is naturally isomorphic to $\Hom_S(\Omega^1_{S/R}, -)$. 
                \end{proof}
            \begin{theorem}[Universal property of K\"ahler differentials] \label{theorem: kahler_differentials_universal_property}
                Let $R$ be a commutative ring. Then, there is an adjunction between $\Omega^1_{-/R}: {}^{R/}\Comm\Alg \to $ 
            \end{theorem}
            
            \begin{remark}[K\"ahler differentials and colimits] \label{remark: differentials_and_colimits}
                This could be viewed as a corollary to theorem \ref{theorem: kahler_differentials_universal_property}. 
                
                Let $R$ be a base commutative ring and let $\{S_i\}_{i \in I}$ be a diagram of $R$-algebras $S_i$. Then, due to $\Omega^1_{-/R}$ being a left-adjoint (which means, in particular, that it would preserve colimits \textit{a priori}), one has the following identity:
                    $$\Omega^1_{\underset{i \in I}{\colim} S_i/R} \cong \underset{i \in I}{\colim} \Omega^1_{S_i/R}$$
            \end{remark}
            \begin{example}[K\"ahler differentials and localisations] \label{example: differentials_and_localisations}
                An example of a colimit of an infinite diagram of modules of K\"ahler differentials is how these modules interact with localisations of commutative rings. Let $S$ be a (possibly infinite) commutative ring let $\q \in |\Spec S|$ be a prime ideal thereof, and let $R \to S$ be a ring map. Then:
                    $$\Omega^1_{S_{\q}/R} \cong (\Omega^1_{S/R})_{\q}$$
                Note that this exhibits the commutativity of $\Omega^1_{-/R}$ with an \textit{infinite} colimit because:
                    $$S_{\q} \cong \underset{y \in S \setminus \q}{\colim} S[1/y]$$
            \end{example}
            Let us examine how K\"ahler differentials interact with colimits a bit closer through the following proposition, wherein we rely on the fact that finite colimits can be constructed out of finite coproducts and epimorphisms.
            \begin{proposition}[K\"ahler differentials and finite colimits] \label{prop: differentials_and_finite_colimits}
                Let $R$ be a base commutative ring. 
                    \begin{enumerate}
                        \item \textbf{(Module of differentials of a surjection):} If $R \to S$ is a surjective ring homomorphism, then:
                            $$\Omega^1_{S/R} \cong 0$$
                        \item \textbf{(Module of differentials and base change):} Consider a pushout diagram of commutative rings such as the following one:
                            $$
                                \begin{tikzcd}
                                	{S'} & {R'} \\
                                	S & R
                                	\arrow[from=2-2, to=2-1]
                                	\arrow[from=2-1, to=1-1]
                                	\arrow[from=2-2, to=1-2]
                                	\arrow[from=1-2, to=1-1]
                                	\arrow["\lrcorner"{anchor=center, pos=0.125}, draw=none, from=1-1, to=2-2]
                                \end{tikzcd}
                            $$
                        Then:
                            $$\Omega^1_{S'/R} \cong \Omega^1_{S/R} \oplus \Omega^1_{R'/R}$$
                    \end{enumerate}
            \end{proposition}
                \begin{proof}
                    \noindent
                    \begin{enumerate}
                        \item \textbf{(Module of differentials of a surjection):} 
                            \begin{enumerate}
                                \item First of all, we claim that $\Omega^1_{R/R} \cong 0$. To see why this is the case, recall firstly that $R$ is the initial object of ${}^{R/}\Comm\Alg$, the category of internal commutative and unital algebras in $R\mod$. The universal property of the initial object as the colimit of the empty diagram as well as the fact that $\Omega^1_{-/R}$ is a left-adjoint, then jointly imply that $\Omega^1_{R/R}$ must be initial in $R\mod$. Lastly, recall that $0$ is intial in $R\mod$: this implies that $\Omega^1_{R/R} \cong 0$. Note that this is a special case of $\Omega^1_{S/R} \cong 0$ whenever $R \to S$ is surjective, because the zero object $0 \in R\mod$ is also terminal (and also because $R\mod$ is an abelian category).
                                \item By remark \ref{remark: differentials_and_colimits}, there exists a surjective $R$-module homomorphism:
                                    $$\Omega^1_{R/R} \to \Omega^1_{S/R}$$
                                and because $\Omega^1_{R/R} \cong 0$, we can thus deduce that:
                                    $$\Omega^1_{S/R} \cong 0$$
                                from the universal property of the zero object $0$ in $R\mod$ as the limit of the empty diagram.
                            \end{enumerate}
                        \item \textbf{(Module of differentials and base change):} This is completely trivial.
                    \end{enumerate}
                \end{proof}
                
            \begin{lemma}[Surjections between modules of differentials] \label{lemma: surjections_between_modules_of_differentials}
                Consider the following commutative diagram in $\Cring$:
                    $$
                        \begin{tikzcd}
                        	{S'} & {R'} \\
                        	S & R
                        	\arrow[from=2-1, to=1-1]
                        	\arrow[from=2-2, to=1-2]
                        	\arrow[from=1-2, to=1-1]
                        	\arrow[from=2-2, to=2-1]
                        \end{tikzcd}
                    $$
                Should the arrow $S \to S'$ be surjective, then the naturally induced $S$-module homomorphism $\Omega^1_{S/R} \to \Omega^1_{S'/R'}$ shall also be surjective.
            \end{lemma}
                \begin{proof}
                    First of all, the $S$-module homomorphism $\Omega^1_{S/R} \to \Omega^1_{S'/R'}$ is well-defined as it comes from evaluating the natural transformation $\Omega^1_{-/R} \to \Omega^1_{-/R'}$ along the arrow $S \to S'$ in the following manner:
                        $$
                            \begin{tikzcd}
                            	& {} & {\Omega^1_{S'/R'}} & 0 \\
                            	{\Omega^1_{S'/R}} & {\Omega^1_{R'/R}} \\
                            	{\Omega^1_{S/R}} & 0
                            	\arrow[from=3-2, to=2-2]
                            	\arrow[from=3-2, to=3-1]
                            	\arrow[from=3-1, to=2-1]
                            	\arrow[from=2-2, to=2-1]
                            	\arrow[from=1-4, to=1-3]
                            	\arrow[from=2-1, to=1-3]
                            	\arrow[from=2-2, to=1-4]
                            \end{tikzcd}
                        $$
                    Next, note that the $S$-module homomorphism $\Omega^1_{S/R} \to \Omega^1_{S'/R}$ is trivially surjective via an application of proposition \ref{prop: differentials_and_finite_colimits}.
                \end{proof}
                
            \begin{proposition}[The canonical exact sequence] \label{prop: canonical_exact_sequence_of_differentials}
                For $A \to B \to C$ a composition of ring maps, there exists a canonically associated right-exact sequence of $C$-modules:
                    $$C \tensor_B \Omega^1_{B/A} \to \Omega^1_{C/A} \to \Omega^1_{C/B} \to 0$$
            \end{proposition}
                \begin{proof}
                    First of all, the morphisms $A \to B \to C$ gives rise to a natural transformations:
                        $$\Omega^1_{-/A} \to \Omega^1_{-/B} \to \Omega^1_{-/C}$$
                    In particular, this tells us that there are the following canonically defined commutative diagrams:
                        $$\Omega^1_{B/A} \to \Omega^1_{B/B}$$
                        $$\Omega^1_{C/A} \to \Omega^1_{C/B} \to \Omega^1_{C/C}$$
                    Second of all, recall that we know by proposition \ref{prop: differentials_and_finite_colimits} that:
                        $$\Omega^1_{B/B} \cong 0$$
                        $$\Omega^1_{C/C} \cong 0$$
                    Thus, there exists the following canonical commutative diagram of $C$-modules:
                        $$
                            \begin{tikzcd}
                            	{C \tensor_B \Omega^1_{B/A}} & {\Omega^1_{C/A}} & 0 \\
                            	0 & {\Omega^1_{C/B}} & 0
                            	\arrow[from=1-2, to=2-2]
                            	\arrow[from=1-1, to=2-1]
                            	\arrow[from=2-1, to=2-2]
                            	\arrow[from=1-1, to=1-2]
                            	\arrow[from=2-2, to=2-3]
                            	\arrow["{!}", from=1-2, to=1-3]
                            	\arrow[from=1-3, to=2-3]
                            \end{tikzcd}
                        $$
                    wherein:
                        \begin{itemize}
                            \item the horizontal arrows exist as a consequence of $C \tensor_B -: B\mod \to C\mod$ being a left-adjoint
                            \item $!: \Omega^1_{C/A} \to 0$ is the canonical terminal arrow, and
                            \item the arrows $0 \to \Omega^1_{C/B}$ and $\Omega^1_{C/B} \to 0$ are actually $C \tensor_B \Omega^1_{B/B} \to \Omega^1_{C/B}$ and $\Omega^1_{C/B} \to \Omega^1_{C/C}$, respectively.
                        \end{itemize}
                    An application of lemma \ref{lemma: surjections_between_modules_of_differentials} to the square:
                        $$
                            \begin{tikzcd}
                            	C & B \\
                            	C & A
                            	\arrow["{\id_C}", from=2-1, to=1-1]
                            	\arrow[from=2-2, to=1-2]
                            	\arrow[from=1-2, to=1-1]
                            	\arrow[from=2-2, to=2-1]
                            \end{tikzcd}
                        $$
                    (note that the identity morphism $\id_C: C \to C$ is trivially surjective) then helps us show the surjectivity of the map $\Omega^1_{C/A} \to \Omega^1_{C/B}$. This concludes the proof.
                \end{proof}
                
        \subsubsection{Cotangent complexes}

    \subsection{Smoothness}
        Smoothness is a notion that, while being intuitively simple (or at least seemingly so), is extremely subtle and furthermore, has far-reaching consequences. Morally, one should imagine a smooth scheme (or for that matter, a smooth variety) as an algebro-geometric object that behaves as much like a smooth manifold as possible. For instance, there ought to be no singularities, as well as no funny business of dimension-hopping between tangent spaces at different points. But what if one is looking for something a bit more technical ? Well, first of all, we are going to restrict ourselves to cases where a so-called \say{smooth} morphism is of finite presentation, which is because our first line of attack is going to be through Jacobian matrices: should these be of full rank, our schemes shall be \say{smooth}, and since Jacobians are only well-defined for functions between finite-dimensional spaces, \say{smooth} morphisms had better be of finite presentation in the first place (otherwise, there might be infinitely many components in our Jacobians). This, however, turns out to be a na\"ive attempt at tackling algebro-geometric smoothness, which is not to imply that one is unable to write down a meaningful definition of what it means for a scheme to be smooth, but instead, that such a definition is entirely impractical (this was pushed, for instance, by Michael Artin): the Jacobian criterion, or even the alternative definition involving the cotangent complex, while concrete, is just not easy to check at all, and worse, does not generalise well to more exotic settings such as those of derived schemes or perfectoid spaces. Due to this, we will start with what is called \say{formal smoothness}. A formally smooth morphism, roughly speaking, shall be one with all the qualitative properties that one would expect from a smooth morphism. We will subsequently introduce finiteness to the picture to obtain morphisms that are smooth in the technical sense. 

        \subsubsection{Formally smooth morphisms}
            \begin{definition}[Formal smoothness] \label{def: formal_smoothness} \index{Smoothness! formal}
                \noindent
                \begin{enumerate}
                    \item \textbf{(Formally smooth ring map):} A homomorphism of commutative rings:
                        $$\varphi: R \to S$$
                    is \textbf{formally smooth} if and only if for all $S$-algebras $B$ and nilpotent ideal $J$ thereof, the canonical map induced by the ring map $B \to B/J$:
                        $$\Spec S(B) \to \Spec R(B/J)$$
                    is surjective.
                    \item \textbf{(Formally smooth prestacks):} A morphism:
                        $$f: \calX \to \calY$$
                    of prestacks is said to be \textbf{formally smooth} if and only if it is represented by a formally smooth morphism of affine schemes.
                \end{enumerate}
            \end{definition}
            \begin{remark}
                Note that the so-called \say{canonical map} induced by $B \to B/J$ always exists; simply consider the following commutative diagram:
                    $$
                        \begin{tikzcd}
                        	{X(B)} & {X(B/J)} \\
                        	{Y(B)} & {Y(B/J)}
                        	\arrow[from=1-1, to=2-1]
                        	\arrow[from=2-1, to=2-2]
                        	\arrow[from=1-1, to=1-2]
                        	\arrow[from=1-2, to=2-2]
                        	\arrow[dashed, from=1-1, to=2-2]
                        \end{tikzcd}
                    $$
            \end{remark}
            
            \begin{proposition}[Formal smoothness is stable under base changes and compositions] \label{prop: compositions_and_base_changes_of_formally_smooth_morphisms}
                \noindent
                \begin{enumerate}
                    \item Let:
                        $$
                            \begin{tikzcd}
                            	A & B & C
                            	\arrow["\varphi", from=1-1, to=1-2]
                            	\arrow["\psi", from=1-2, to=1-3]
                            \end{tikzcd}
                        $$
                    be a composition of formally smooth ring homomorphisms. The composite map $A \to C$ is thus also formally smooth.
                    \item Let $\varphi: R \to S$ be a formally smooth ring map and $\psi: R \to R'$ be an arbitrary homomorphism of commutative rings. Then, the pushout $S \tensor_{\varphi, R, \psi} R'$ is formally smooth over $R'$ as well.
                \end{enumerate}
            \end{proposition}
                \begin{proof}
                    \noindent
                    \begin{enumerate}
                        \item 
                        \item 
                    \end{enumerate}
                \end{proof}
                
            \begin{lemma}[Splitting of the canonical short exact sequence] \label{lemma: canonical_short_exact_sequence_splits}
                Let $\varphi: R \to S$ be a ring map and let $\pi: P \to S$ be a surjective homomorphism of $R$-algebras from a polynomial $R$-algebra $P$; additionally, write $J := \ker \pi$. Then, $\varphi: R \to S$ is smooth if and only if the canonically defined right-exact sequence:
                    $$J/J^2 \to \Omega^1_{P/R} \tensor_P S \to \Omega^1_{S/R} \to 0$$
                is actually a short exact sequence that splits.
            \end{lemma}
                \begin{proof}
                    \noindent
                    \begin{enumerate}
                        \item 
                        \item 
                    \end{enumerate}
                \end{proof}
            
            \begin{proposition}
                
            \end{proposition}
                \begin{proof}
                    
                \end{proof}
                
            \begin{proposition}[Formal smoothness is a local property] \label{prop: formal_smoothness_is_local}
                Let $\varphi: R \to S$ be a homomorphism between two commutative rings and let $\q$ be some prime ideal of $S$ (read: point of $|\Spec S|$). Then, $\varphi$ is formally smooth if and only if the induced maps $\varphi_{\q}: R \to S_{\q}$ are all formally smooth. 
            \end{proposition}
                \begin{proof}
                    \noindent
                    \begin{enumerate}
                        \item Suppose first of all that $\varphi_{\q}: R \to S_{\q}$ is a formally smooth ring map for any prime $\q \in |\Spec S|$.  
                        \item 
                    \end{enumerate}
                \end{proof}
                
            \begin{proposition}[Formally smooth + finite type + local = flat] \label{prop: formally_smooth_finite_type_local_morphisms_are_flat}
                Let $(R, \m)$ be a local ring, let $S$ be a finitely presented $R$-algebra, and consider a local homomorphism $(R, \m) \to (S_{\q}, \q)$. Then, should $R \to S_{\q}$ be formally smooth, it shall also be flat. 
            \end{proposition}
                \begin{proof}
                    
                \end{proof}
    
        \subsubsection{Smooth morphisms}
            \begin{definition}[Standard smoothness] \label{def: standard_smoothness} \index{Smoothness! standard}
                \noindent
                \begin{enumerate}
                    \item \textbf{(Standard smooth ring maps):} A map of commutative rings:
                        $$\varphi: R \to S$$
                    is called \textbf{standard smooth} if and only if it is of \textit{finite presentation} (i.e. there exists natural numbers $N, n$ such that:
                        $$S \cong R[x_1, ..., x_N]/(f_1, ..., f_n)$$
                    for some finite subset $\{f_i\}_{1 \leq i \leq n}$ of $R[x_1, ..., x_n]$) and the Jacobian of the vector-valued function $(f_1, ..., f_n)$ (mind the abuse of notation):
                        $$\Jac(f_1, ..., f_n) = \left(\nabla f_1, ..., \nabla f_n\right)^T = 
                            \begin{pmatrix}
                                \del_{x_1} f_1 & ... & \del_{x_n} f_1
                                \\
                                \vdots & \ddots & \vdots
                                \\
                                \del_{x_1} f_n & ... & \del_{x_n} f_n
                            \end{pmatrix}
                        = (\del_{x_j} f_i)_{1 \leq i, j \leq n}$$
                    is \textit{full-rank} (i.e. of rank $n$ in this particular instance); alternatively, by basic module theory, one can require the determinant of the Jacobian to be \textit{invertible} in $S$.  
                    \item \textbf{(Standard smooth prestacks):} A morphism:
                        $$f: \calX \to \calY$$
                    of prestacks is said to be \textbf{standard smooth} if and only if it is represented by a standard smooth morphism of affine schemes.
                \end{enumerate}
            \end{definition}
            \begin{remark}[Unpacking the definition] \label{remark: standard_smoothness}
                Definition \ref{def: standard_smoothness} paints a rather conrete and down-to-earth picture depicting what it means for a ring map to supposedly be \say{smooth}. Essentially, what it is trying to say is that given a ring map of finite presentation:
                    $$\varphi: R \to S$$
                with:
                    $$S \cong R[x_1, ..., x_N]/(f_1, ..., f_n)$$
                then should the Jacobian - an $R$-linear operator on $S$ viewed as a finitely presented $R$-module - be of full rank, the aforementioned ring map $\varphi$ is going to be somehow \say{smooth} (the quotation marks are here because as it turns out, standard smooth morphisms are only cohomologically smooth - i.e. smooth in the \say{right} algebro-geometric way - if the associated universal module of K\"ahler differential is free; cf. proposition \ref{prop: smooth_iff_standard_smooth}). In other words, definition \ref{def: standard_smoothness} is nothing but an analogue of the Inverse Function Theorem from calculus. 
            \end{remark}
            \begin{remark}[Locality of (standard) smoothness]
                One very important bit of information that can be inferred from definition \ref{def: standard_smoothness} is that standard smoothness (and as we shall see later on, cohomological smoothness as well) is a Zariski-local property: one checks whether or not some given scheme over a base commutative ring is standard smooth by checking if the affine patches covering it are so. 
            \end{remark}
            
            \begin{definition}[Cohomological smoothness] \label{def: cohomological_smoothness} \index{Smoothness! cohomological}
                \noindent
                \begin{enumerate}
                    \item \textbf{(Cohomologically smooth ring maps):} A homomorphism between commutative rings:
                        $$\varphi: R \to S$$
                    is called \textbf{cohomologically smooth} if and only if it is of finite presentation and its associated (na\"ive) cotangent complex is quasi-isomorphic to a finitely generated projective $S$-module placed in degree $0$.
                    \item \textbf{(Cohomologically smooth prestacks):} A morphism:
                        $$f: \calX \to \calY$$
                    of prestacks is said to be \textbf{cohomologically smooth} if and only if it is represented by a cohomologically smooth morphism of affine schemes.
                \end{enumerate}
            \end{definition}
            \begin{remark}[Cotangent complex: na\"ive or nay ?]
                Definition \ref{def: cohomological_smoothness} made reference to na\"ive cotangent complexes associated to ring maps, and how those of ring maps that are of finite presentation being quasi-isomorphic to certain complexes of modules concentrated in degree $0$ implies cohomological smoothness. On the surface this might seem like a rather sensible characterisation of smoothness, but dive a little deeper and one shall find one glaring problem: the na\"ive cotangent complex is incredibly awkward to work with. There is, however, a silver lining, which is that na\"ive cotangent complexes are actually nothing but $(-1)$-truncated cotangent complexes. Thus, we can simply remove the word \say{na\"ive} from definition \ref{def: cohomological_smoothness}. 
            \end{remark}
            
            \begin{proposition}[Cohomological smoothness is the same as standard smoothness] \label{prop: smooth_iff_standard_smooth}
                A ring map $\varphi: R \to S$ of finite presentation is smooth if and only if it is standard smooth.
            \end{proposition}
                \begin{proof}
                
                \end{proof} 
            \begin{convention}
                Thanks to proposition \ref{prop: smooth_iff_standard_smooth}, it makes sense from this point on for us to do away with the specifications and refer to both standard smooth morphisms and cohomologically smooth ones as simply being \say{smooth}.
            \end{convention}
                
            \begin{proposition}[Smoothness implies almost-finiteness of cotangent complex] \label{prop: smoothness_implies_almost_finiteness_of_cotangent_complex}
                The cotangent complex associated to any smooth ring map $\varphi: R \to S$ is almost of finite type, and because the cotangent complex associated to any smooth ring map is quasi-isomorphic to a projective module placed in degree $0$, this is actually just asserting that the associated module of K\"ahler differentials $\Omega^1_{S/R}$ is a finitely generated projective module.
            \end{proposition}
                \begin{proof}
                
                \end{proof}
            \begin{corollary}[Relative dimensions of smooth maps]
                The relative dimension of a smooth ring map is the number of generators of its associated cotangent complex, which according to proposition \ref{prop: smoothness_implies_almost_finiteness_of_cotangent_complex}, had better be finite.
            \end{corollary}
            \begin{example}
                A smooth ring map of the form:
                    $$\varphi: R \to R[x_1, ..., x_N]/(f_1, ..., f_n)$$
                has relative dimension $N - n$. 
            \end{example}
            
            \begin{proposition}[Smooth maps are finitely presented formally smooth maps] \label{prop: smooth_iff_formally_smooth_and_of_finite_presentation}
                A ring map of finite presentation is smooth if and only if it is formally smooth.
            \end{proposition}
                \begin{proof}
                    
                \end{proof}
            
            \begin{proposition}[Smoothness is a local property] \label{prop: smoothness_is_local}
                Let $\varphi: R \to S$ be a ring map of finite presentation and let $\q$ be some prime ideal of $S$ (read: point of $|\Spec S|$). Then, $\varphi$ is smooth if and only if the induced maps $\varphi_{\q}: R \to S_{\q}$ are all smooth. 
            \end{proposition}
                \begin{proof}
                    
                \end{proof}
            \begin{corollary}[Fibre-wise smoothness] \label{coro: fibrewise_smoothness}
                Let $X$ be a scheme over some base scheme $S$. Then, the structure morphism $X \to S$ is smooth if and only if all of its fibres are so, i.e. for all $s \in |S|$, the fibre $X_s \cong X \x_S \Spec \kappa_s$ is smooth over the residue field $\kappa_s$. In practice, this means that to check for smoothness, one can simply pullback to over a point and apply fibre-wise results on smoothness (such as proposition \ref{prop: dimensions_of_smoothn_morphisms_over_fields}).
            \end{corollary}
                
            \begin{proposition}[Smoothness is stable under base changes and compositions] \label{prop: compositions_and_base_changes_of_smooth_morphisms}
                \noindent
                \begin{enumerate}
                    \item Let:
                        $$
                            \begin{tikzcd}
                            	A & B & C
                            	\arrow["\varphi", from=1-1, to=1-2]
                            	\arrow["\psi", from=1-2, to=1-3]
                            \end{tikzcd}
                        $$
                    be a composition of smooth ring homomorphisms, and suppose that $\varphi$ is of relative dimension $r$, and $\psi$ is of relative dimension $s$. Given these hypotheses, the relative dimension of $\psi \circ \varphi$ is $r + s$.
                    \item Let $\varphi: R \to S$ be a smooth ring map of relative dimension $d$ and $\psi: R \to R'$ be an arbitrary homomorphism of commutative rings. Then, the pushout $S \tensor_{\varphi, R, \psi} R'$ is smooth over $R'$, and of relative dimension $d$ as well. 
                \end{enumerate}
            \end{proposition}
                \begin{proof}
                    \noindent
                    \begin{enumerate}
                        \item According to definition \ref{def: standard_smoothness} and proposition \ref{prop: smooth_iff_standard_smooth}, we can write $B$ as a commutative $A$-algebra of the form $\frac{A[x_1, ..., x_N]}{(f_1, ..., f_n)}$ for some pair $N, n$ of natural numbers, and subsequently, $C$ as a commutative $B$-algebra (which should be viewed as an $\frac{A[x_1, ..., x_N]}{(f_1, ..., f_n)}$-algebra) of the form $\frac{\frac{A[x_1, ..., x_N]}{(f_1, ..., f_n)}[y_1, ..., y_M]}{(g_1, ..., g_m)} \cong \frac{A[x_1, ..., x_N, y_1, ..., y_M]}{(f_1, ..., f_n, g_1, ..., g_m)}$ for another pair $M, m$ of natural numbers. Notice that:
                            $$N - n = r, M - m = s$$
                        (also, recall that smooth morphisms are \textit{a priori} of finite presentation, which would imply that $n \leq N$ and $m \leq M$, so the above expressions are well-defined - we do not want negative dimensions, after all). It is then rather easy to see that the relative dimension of $\psi \circ \varphi$ had better be equal to $r + s$.
                        \item Suppose that for some pair of natural numbers $n, N$, we have:
                            $$S \cong \frac{R[x_1, ..., x_N]}{(f_1, ..., f_n)}$$
                        Then, by the fact that colimits commute, we have:
                            $$S \tensor_{\varphi, R, \psi} R' \cong \frac{R[x_1, ..., x_N]}{(f_1, ..., f_n)} \tensor_{\varphi, R, \psi} R' \cong \frac{R'[x_1, ..., x_N]}{(f_1, ..., f_n)}$$
                        which tells us that the pushout $S \tensor_{\varphi, R, \psi} R'$ is smooth as a commutative $R'$-algebra, and that it is of relative dimension $d = N - n$, much like $S$ is as an $R$-algebra.
                    \end{enumerate}
                \end{proof}
            \begin{remark}[Preservation of smoothness and \'etale-ness of non-affine schemes]
                As smoothness is a local notion (cf. proposition \ref{prop: smoothness_is_local}), and as such proposition \ref{prop: compositions_and_base_changes_of_smooth_morphisms} generalises in a rather obvious manner to cases where one's schemes might not be affine. Namely:
                    \begin{enumerate}
                        \item should:
                            $$
                                \begin{tikzcd}
                                    	X & Y & Z
                                    	\arrow["\varphi", from=1-1, to=1-2]
                                    	\arrow["\psi", from=1-2, to=1-3]
                                    \end{tikzcd}
                            $$
                        be any pair of composable smooth (or \'etale) morphisms of schemes, wherein $\varphi$ is of relative dimension $r$ and $\psi$ is of relative dimension $s$, then their composition $\psi \circ \varphi$ will be smooth and of relative dimension $r + s$, and
                        \item given any pullback square of schemes as follows:
                            $$
                                \begin{tikzcd}
                                	{Y'} & Y \\
                                	{X'} & X
                                	\arrow["\psi", from=2-1, to=2-2]
                                	\arrow["\varphi", from=1-2, to=2-2]
                                	\arrow[from=1-1, to=1-2]
                                	\arrow[from=1-1, to=2-1]
                                	\arrow["\lrcorner"{anchor=center, pos=0.125}, draw=none, from=1-1, to=2-2]
                                \end{tikzcd}
                            $$
                        wherein $\varphi: Y \to X$ is smooth of relative dimension $d$ and $\psi: X' \to X$ is arbitrary, the canonical projection $Y \x_{\varphi, X, \psi} X' \to X'$ is also smooth and of relative dimension $d$ (when $d = 0$, one obtains the stability of \'etale-ness of under base changes).
                    \end{enumerate}
            \end{remark}
            
            \begin{proposition}[Relative and pure dimensions of smooth maps over fields] \label{prop: dimensions_of_smoothn_morphisms_over_fields}
                Let $k$ be a field and let:
                    $$\pi: X \to \Spec k$$
                be a scheme that is smooth over $\Spec k$. Then, the following are equivalent:
                    \begin{enumerate}
                        \item $\pi: X \to \Spec k$ is of relative dimension $d$. 
                        \item The Krull dimension of $X$ is $d$. 
                    \end{enumerate}
            \end{proposition}
                \begin{proof}
                    Because smoothness, as a property of schemes, is Zariski-local, let us assume that $X$ is affine. Note that this is not at the detriment of generality. 
                    \begin{enumerate}
                        \item To start, let us assume \textbf{1}. Specifically, let us assume that for some pair of natural numbers $n, N$ such that $d = N - n$, we have:
                            $$X \cong \Spec \frac{k[x_1, ..., x_N]}{(f_1, ...,f_n)}$$
                        Then, it is simply a matter of finding the Krull dimension $\frac{k[x_1, ..., x_N]}{(f_1, ...,f_n)}$. By the Third Isomorphism Theorem, prime ideals of $\frac{k[x_1, ..., x_N]}{(f_1, ...,f_n)}$ are in bijective correspondence with those of $k[x_1, ..., x_n]$ that contain the ideal $(f_1, ..., f_n)$; the Krull dimension of $\frac{k[x_1, ..., x_N]}{(f_1, ...,f_n)}$ is thus, by definition, the supremum of the heights of such prime ideals. 
                        \item Let:
                            $$X \cong \Spec \frac{k[x_1, ..., x_N]}{(f_1, ..., f_n)}$$
                        and suppose that:
                            $$\dim_{\Krull} X = d$$
                        
                    \end{enumerate}
                \end{proof}