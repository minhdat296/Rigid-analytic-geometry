\section{Classical rigid-analytic spaces}
    \begin{convention} \label{conv: rigid_analytic_varieties_non_archimedean_ground_field}
        Henceforth, we fix a \textit{complete}\footnote{This is an important hypothesis because eventually, we will want out affinoid algebras to be instances of Banach algebra over $K$ (cf. proposition \ref{prop: affinoid_algebras_are_banach_algebras}).} non-archimedean field $(K, |-|)$ valuation ring $K^{\circ} := \{x \in K \mid |x| \leq 1\}$ and residue field $k := K^{\circ}/\m_K$, where $\m_K := \{x \in K \mid |x| < 1\}$ is the unique maximal ideal of $K^{\circ}$. In addition, we will be caring about the value group of $K$ (or rather, of $K^{\circ}$), which comes from the valuation $v: K^{\x} \to \R$ that determines the non-archimedean ultra-metric $|-| := \exp(v(-))$, and is given by $\Gamma := \im v$ (note that this is well-defined because $v: K^{\x} \to \R$ is a group homomorphism); this is a totally ordered abelian group whose ordering is inherited from the archimedean ordering of $(\R, \leq)$.
    \end{convention}

    \subsection{Tate's rigid spaces}
        \subsubsection{Affinoid algebras}
            \begin{remark}
                In addition, recall that the valuation ring $K^{\circ}$ is Noetherian if and only if the value group $\Gamma$ is a discrete subgroup of $\R$; in which case, $K^{\circ}$ will be a discrete valuation ring.
            \end{remark}
            
            \begin{convention}[Multi-indices] \label{conv: mutli_indices}
                Fix a $n \in \N$. An \textbf{$n$-multi-index} is an ordered $n$-tuple $J := (j_1, ..., j_n) \in \N^n$. By $|J|$, we shall mean $\sum_{i = 1}^n j_i$ and for any ordered $n$-tuple of variables $X := (x_1, ..., x_n)$ we shall write $X^J := \prod_{i = 1}^n x_i^{j_i}$.
            \end{convention}
            \begin{definition}[Tate algebras and affinoid algebras] \label{def: affinoid_algebras}
                Let $n \in \N$ be a fixed natural number and let $X := (x_1, ..., x_n)$ be an $n$-tuple of variables ($0$-tuples are to be understood as the empty set). The \textbf{$n^{th}$ Tate $K$-algebra}, denoted by $K\{X\} := K\{x_1, ..., x_n\}$, is the $K$-subalgebra\footnote{We will let the readers check for themselves that $K\{X\}$ is a $K$-subalgebra of $K[\![X]\!]$.} of $K[\![X]\!] := K[\![x_1, ..., x_n]\!]$ whose elements are the convergent\footnote{Recall that a non-archimedean power series $\sum_{|J| = 0}^{+\infty} a_J X^J$ with coefficients $a_J \in K$ converges if and only if the coefficients $a_J$ are topologically nilpotent.} power series $\sum_{|J| = 1}^{+\infty} a_J X^J$ with coefficients $a_J \in K$; when $n = 0$, we get the zero ring $K\{\varnothing\} := 0$. This is naturally a normed algebra via the so-called \textbf{Gauss norm}, which is given by $\left\|\sum_{|J| = 0}^{+\infty} a_J X^J\right\| := \sup_J |a_J|$ and well-defined because non-archimedean convergent power series must have topologically nilpotent (hence bounded) coefficients; obviously $\|f\| = 0$ if and only if $f = 0$.
                
                An \textbf{affinoid $K$-algebra} is either the zero ring or a quotient of a Tate $K$-algebra by an ideal thereof (i.e. Tate algebras are \say{free} affinoid algebras). 
            \end{definition}
            
            Before we move on, let us first recall several aspects of the theory of Banach spaces over non-archimedean fields, mostly to be clear about what we specfically mean by such \say{Banach spaces}.
            \begin{definition}[Non-archimedean Banach spaces] \label{def: non_archimedean_banach_spaces}
                A \textbf{non-archimedean $K$-vector space} is a $K$-vector space equipped with an ultra-norm. A \textbf{Banach space} over $K$ is a topologically complete non-archimedean $K$-vector space.
            \end{definition}
            \begin{definition}[Bounded linear maps] \label{def: bounded_linear_maps_between_non_archimedean_vector_spaces}
                A $K$-linear map $\varphi: V \to W$ between two non-archimedean $K$-vector spaces $V, W$ is said to be \textbf{bounded} if and only if there exists $C \geq 0$ such that $\|\varphi(v)\| \leq C \|v\|$ for all $v \in V$.
            \end{definition}
            \begin{remark}
                It is easy to check that the identity map on any non-archimedean $K$-vector space is bounded, and furthermore, that the boundedness of $K$-linear maps between non-archimedean $K$-vector spaces preserved under compositions.
            \end{remark}
            \begin{proposition}
                A $K$-linear map between two non-archimedean $K$-vector spaces is bounded if and only if it is continuous with respect to the topologies induced by the ultra-norms on the codomain and domain.
            \end{proposition}
                \begin{proof}
                    
                \end{proof}
            \begin{corollary}
                There is a category of (non-archimedean) $K$-Banach spaces, which we denote by $K\Ban$, whose objects are $K$-Banach spaces and whose morphisms are the continuous (i.e. bounded) $K$-linear maps.
            \end{corollary}
            \begin{lemma}[(Co)kernels of non-archimedean Banach spaces] \label{lemma: (co)kernels_of_non_archimedean_banach_spaces}
                The category $K\Ban$ wherein the objects are non-archimedean $K$-Banach spaces and the morphisms are the continuous (i.e. bounded) $K$-linear maps is abelian, small-complete, and small cocomplete, wherein zero objects are isomorphic to the zero $K$-Banach space $0$. In particular, one equips subspaces of non-archimedean $K$-Banach spaces with the subspace topology and quotients (say, $V/V_0$ for $V_0 \subseteq V$ some closed\footnote{Recall that open subspaces of topological vector spaces are actually also closed (cf. \ref{lemma: open_submodules_are_closed}).} subspace) thereof with the norm given for every $x \in V/V_0$ by:
                    $$\|x\|_{V/V_0} := \inf_{v \equiv x \pmod{V_0}} \|v\|_V$$
            \end{lemma}
                \begin{proof}
                    
                \end{proof}
            \begin{remark}
                When $K$ is replaced by some arbitrary non-archimedean Banach ring $R$, one loses certain nice properties. In particular, $R\Ban$ fails to either be $AB5$ or $AB5^*$: that is, it is not guaranteed that filtered (co)limits are exact. Of course, this is always guaranteed for $K\Ban$, since vector spaces are free, hence flat.
            \end{remark}
            \begin{definition}[Completed tensor products] \label{def: completed_tensor_products_of_non_archimedean_vector_spaces}
                The \textbf{completed tensor product} (also known as the \textbf{projective tensor product}) $V \hat\tensor_K W$ of two non-archimedean $K$-vector spaces $V$ and $W$ is the topological completion of the algebraic tensor product $V \tensor_K W$ with respect to the so-called \textbf{projective cross ultra-norm}, which for elements $x := \sum_{i = 1}^n v_i \tensor w_i \in V \tensor_K W$ is given by:
                    $$\|x\|_{\pi} := \inf\left\{ \max_{1 \leq i \leq n} \|v_i\|_V \|w_i\|_W \right\}$$
            \end{definition}
            \begin{remark}
                Of course, one ought to check that the projective cross ultra-norm as in definition \ref{def: completed_tensor_products_of_non_archimedean_vector_spaces} is indeed a well-defined ultra-norm. This, however, is easy and as such will be left to our readers as an exercise.  
            \end{remark}
            \begin{proposition}[Completed tensor products of non-archimedean Banach spaces] \label{prop: completed_tensor_products_of_non_archimedean_banach_spaces}
                The category $K\Ban$ is an infinite tensor category\footnote{This is as in \cite[Definition 4.1.1]{EGNO}, but without the local finiteness condition. In particular, this means that $K\Ban$ is symmetric monoidal, rigid, and closed with respect to the completed tensor product of non-archimedean $K$-vector spaces.}, whose monoidal structure is the completed tensor product $\hattensor_K$ of non-archimedean $K$-vector spaces.
            \end{proposition}
                \begin{proof}
                    
                \end{proof}
            \begin{corollary}
                There exists a category of commutative Banach $K$-algebras internal to $K\Ban$, whose objects are the commutative monoids internal to $K\Ban$ and whose morphisms are the continuous (i.e. bounded) $K$-algebra homomorphisms between them. This category shall be denoted by $K\Ban\Comm\Alg$. This category is complete and cocomplete, and in particular, coproducts are given by the completed tensor product $\hattensor_K$, and initial objects are isomorphic to the zero ring $0$.
            \end{corollary}
                \begin{proof}
                    
                \end{proof}
            
            \begin{lemma}[Ideals of Tate algebras are closed] \label{lemma: tate_algebra_ideals_are_closed}
                Ideals of Tate $K$-algebras are closed with respect to the topology induced by the Gauss norm.
            \end{lemma}
                \begin{proof}
                    
                \end{proof}
            \begin{remark}[Maximal ideals of Banach algebras are closed] \label{remark: maximal_ideals_of_banach_algebras_are_closed}
                Using a method very similar to that used in the proof of lemma \ref{lemma: invertible_elements_of_banach_algebras_form_open_subsets}, we can show that in a (non-archimedean or archimedean) Banach ring, the closure of any non-trivial ideal is also a non-trivial ideal; this, in turn, implies that maximal ideals are closed. Arbitrary ideals, however, are not closed \textit{a priori}: for example, for $X$ a compact Hausdorff topological space, the ideal $C^0_c(X) \subset C^0(X)$ of compactly supported continuous complex-valued functions on $X$ is indeed not closed, since the limit of an arbitrary Cauchy sequence of compactly supported complex-valued functions on $X$ might fail to be compactly supported. 
            \end{remark}
            \begin{proposition}[Affinoid algebras are Banach algebras] \label{prop: affinoid_algebras_are_banach_algebras}
                Affinoid algebras are Banach $K$-algebras when equipped with the Gauss norm (note that this relies crucially on the assumption that $K$ is complete as a non-archimedean field).
            \end{proposition}
                \begin{proof}
                    Thanks to lemma \ref{lemma: (co)kernels_of_non_archimedean_banach_spaces} and \ref{lemma: tate_algebra_ideals_are_closed}, it suffices to only prove that any given Tate algebra $K\{x_1, ..., x_n\}$ (for any $n \in \N$) is a Banach $K$-algebra with respect to the topology induced by the Gauss norm; in fact, because $K\{x_1, ..., x_n\} \cong K\{x_1\} \hattensor_K ... \hattensor_K K\{x_n\}$. For this, pick a Cauchy sequence $\{f_i\}_{i \in \N}$ of single-variable convergent power series $f_i(x) := \sum_{d = 0}^{+\infty} a_d^{(i)} x^d \in K\{x\}$ and observe that because:
                        $$\|f_i(x) - f_j(x)\| = \left\| \sum_{d = 0}^{+\infty} (a_d^{(i)} - a_d^{(j)}) x^d \right\| = \sup_d \|a_d^{(i)} - a_d^{(i)}\|$$
                    the sequence of coefficients $\{a_d^{(i)}\}_{i \in \N}$ is also Cauchy at each degree $d \geq 0$ as a result of the assumption that $\{f_i\}_{i \in \N}$ is Cauchy. Because $K$ is complete by hypothesis, the sequences of coefficients $\{a_d^{(i)}\}_{i \in \N}$ all converge by virtue of being Cauchy. Now, observe that:
                        $$\underset{i \to +\infty}{\lim} f_i(x) = \underset{i \to +\infty}{\lim} \sum_{d = 0}^{+\infty} a_d^{(i)} x^d = \sum_{d = 0}^{+\infty} \left( \underset{i \to +\infty}{\lim} a_d^{(i)} \right) x^d$$
                    per some elementary non-archimedean analysis. As such, the sequence $\{f_i\}_{i \in \N}$ is convergent, and since it is an arbitrary Cauchy sequence, means that $K\{x\}$ (and by extension, any Tate $K$-algebra $K\{x_1, ..., x_n\}$) is complete with respect to the topology induced by the Gauss norm. 
                \end{proof}
            \begin{corollary}
                There is a full subcategory of the category $K\Ban\Comm\Alg$ of commutative Banach $K$-algebras and continuous $K$-algebra homomorphisms, wherein the objects are affinoid $K$-algebras. This category shall be denoted by $K\-\Affd$.
                
                The category $K\-\Affd$ of affinoid $K$-algebras, as a full subcategory of $K\Ban\Comm\Alg$, is closed under all finite limits and finite colimits (taken within $K\Ban\Comm\Alg$).
            \end{corollary}
            
            \begin{lemma}[Power-bounded subrings are Banach] \label{lemma: power_bounded_subrings_of_affinoid_algebras_are_banach}
                For any affinoid $K$-algebra $A$, the subring $A^{\circ} \subset A$ of power-bounded elements is a Banach $K^{\circ}$-algebra.
            \end{lemma}
            \begin{proposition}[Unit discs are mapped to unit discs] \label{prop: unit_discs_are_mapped_to_unit_discs}
                There is a functor:
                    $$(-)^{\circ}: K\Ban\Comm\Alg \to K^{\circ}\Ban\Comm\Alg$$
                which assigns to each affinoid $K$-algebra $A$ the Banach $K^{\circ}$-algebra $A^{\circ}$ of power-bounded elements. 
            \end{proposition}
                \begin{proof}
                    
                \end{proof}
            \begin{proposition}[Tate algebras are free affinoid algebras] \label{prop: tate_algebras_are_free_affinoid_algebras}
                For a fixed natural number $n \in \N$, the Tate $K$-algebra $K\{x_1, ..., x_n\}$ is initial among all Banach $K$-algebras $A$ such that 
            \end{proposition}
                \begin{proof}
                    
                \end{proof}
            \begin{corollary}[Recentering is an automorphism] \label{coro: recentering_tate_algebras_over_non_archimedean_fields_is_an_automorphism}
                For any fixed positve integer $n$ and any $n$-tuple $(c_1, ..., c_n)$ of elements $c_i \in K$, there is an continuous $K$-algebra automorphism on $K\{x_1, ..., x_n\}$ given by $f(x_1, ..., x_n) \mapsto f(x_1 - c_1, ..., x_n - c_n)$.
            \end{corollary}
            
            Let us now move on to some important dimension-theoretic properties of affinoid algebras. The take-home point here is that the maximal spectrum $\Spm A$ of any affinoid $K$-algebra $A$ is, in a sense, \say{analytically quasi-compact and Hausdorff} due to there being a non-archimedean version of the Maximum Modulus Principal on this space (cf. theorem \ref{theorem: maximum_modulus_principle_for_affinoid_algebras}).
            \begin{lemma}[Tate algebras are topologically of finite type] \label{lemma: tate_algebras_are_topologically_of_finite_type}
                For any non-archimedean Banach ring $R$, the polynomial $R$-algebra $R[x]$ is a dense\footnote{In other words, $R\{x\}$ is topologically of finite type as a $K$-algebra.} $R$-subalgebra of $R\{x\}$.
            \end{lemma}
                \begin{proof}
                    This is obvious from the construction of the Gauss norm.
                \end{proof}
            \begin{theorem}[Affinoid algebras are Noetherian] \label{theorem: affinoid_algebras_are_noetherian}
                Any affinoid $K$-algebra is Noetherian\footnote{Notice that this is an affinoid analogue of Hilbert's Basis Theorem for polynomial rings over fields.}.
            \end{theorem}
                \begin{proof}
                
                \end{proof}
            \begin{lemma}[Tate $K$-algebras are regular] \label{lemma: tate_algebras_over_non_archimedean_fields_are_regular}
                For any fixed $n \in \N$, the Tate $K$-algebra $K\{x_1, ..., x_n\}$ is a regular $K$-algebra of Krull equidimension $n$. 
            \end{lemma}
                \begin{proof}
                    
                \end{proof}
            \begin{corollary}
                Suppose that we have an affinoid $K$-algebra $A$ that is also an integral domain. Then $\height \m = \dim A$ for all $\m \in \Spm A$.
            \end{corollary}
                \begin{proof}
                    
                \end{proof}
            \begin{corollary}[Tate $K$-algebras are UFDs] \label{coro: tate_algebras_over_non_archimedean_fields_are_UFDs}
                Tate $K$-algebras are UFDs.
            \end{corollary}
                \begin{proof}
                    
                \end{proof}
            \begin{proposition}[Affinoid algebras admit Noetherian normalisations] \label{prop: affinoid_algebras_admit_noetherian_normalisation}
                Let $A$ be an affinoid $K$-algebra of dimension $d$ (which is necessarily finite, since $A$ is Noetherian). Then there exists a finite $K$-algebra extension $K\{x_1, ..., x_d\} \subset A$.
            \end{proposition}
                \begin{proof}
                    
                \end{proof}
            \begin{corollary}
                Let $A$ be an affinoid $K$-algebra. Then, any finite (commutative) $A$-algebra will also be an affinoid $K$-algebra. 
            \end{corollary}
                \begin{proof}
                    
                \end{proof}
                
            \begin{theorem}[The Maximum Modulus Principal for affinoid algebras] \label{theorem: maximum_modulus_principle_for_affinoid_algebras}
                
            \end{theorem}
                \begin{proof}
                    
                \end{proof}
        
        \subsubsection{The Tate topology and structure sheaves of Tate's rigid spaces}
            \begin{proposition}[Maximal spectra of affinoid algebras] \label{prop: maximal_spectra_of_affinoid_algebras}
                There is a functor:
                    $$\Spm: K\-\Affd^{\op} \to \Sets$$
                which associates to each affinoid $K$-algebra $A$ the set $\Spm A$ whose elements are the maximal ideals of $A$. 
            \end{proposition}
                \begin{proof}
                    
                \end{proof}
            
        \subsubsection{Morphisms and (co)limits of Tate's rigid spaces}
        
        \subsubsection{Coherent sheaves on Tate's rigid spaces}
            It turns out that affinoid $K$-algebras behave remarkably similarly to finite-dimensional polynomial rings over algebraically closed fields (in particular, there is an $K$-affinoid analogue of Hilbert's \textit{Nullstellensatz}\footnote{Tate's \textit{Nullstellensatz} ?}), but at the same time, affinoid $K$-algebras have many more (often non-rational) points. 
            \begin{lemma}[Tate algebras are Jacobson] \label{lemma: tate_algebras_over_non_archimedean_fields_are_jacobson}
                Any affinoid $K$-algebra is Jacobson; that is, every affindoid $K$-algebra $A$ is such that every radical $A$-ideal $I$ is equal to the intersection of the maximal $A$-ideals containing it, i.e. we have $I = \bigcap_{\m \in \Spm A/I} \m$ for every radical $A$-ideal $I$.
            \end{lemma}
                \begin{proof}
                    
                \end{proof}
            \begin{theorem}[Tate's \textit{Nullstellensatz}] \label{theorem: tate_nullstellensatz}
                Let $A$ be an affinoid $K$-algebra. Then for all maximal ideals $\m \in \Spm A$, the corresponding residue field $\kappa(\m) := A_{\m}/\m A_{\m}$ is of finite degree over $K$. 
            \end{theorem}
                \begin{proof}
                    
                \end{proof}   
            \begin{corollary}
                Let $A$ be an affinoid $K$-algebra. Then for all maximal ideals $\m \in \Spm A$, the extension $\kappa(\m)/K$ is algebraic by virtue of being finite.
            \end{corollary}
        
            \begin{theorem}[Tate's Acyclcity Theorem] \label{theorem: tate_acyclicity_theorem_for_rigid_spaces}
                
            \end{theorem}
                \begin{proof}
                    
                \end{proof}
    
    \subsection{Raynaud's formal models for rigid spaces}
        \subsubsection{Coherent sheaves on formal schemes topologically of finite presentations}
            \begin{remark}[Affinoids over complete valuation rings] \label{remark: affinoids_over_complete_valuation_rings}
                Since $(K, |-|)$ (as in convention \ref{conv: rigid_analytic_varieties_non_archimedean_ground_field}) is topologically complete with respect to the topology induced by the ultra-metric $|-|$, and since $K^{\circ}$ is a closed topological subspace, it is also topologically complete with respect to $|-|$. As such, many functional-analytic results concerning affinoid $K$-algebras carry over \textit{verbatim} to so-called \textbf{affinoid $K^{\circ}$-algebras}: these are quotients of \textbf{Tate $K^{\circ}$-algebras}, i.e. they are of the form $K^{\circ}\{x_1, ..., x_n\}/I$ for some $n \in \N$ and some $K^{\circ}\{x_1, ..., x_n\}$-ideal $I$.
                
                In particular, this means that:
                    \begin{itemize}
                        \item there is an infinite tensor category $K^{\circ}\Ban$ of Banach (non-archimedean) $K^{\circ}$-modules and continuous (i.e. bounded\footnote{In fact, bounded by $1$.}) $K^{\circ}$-linear maps where the monoidal structure is given by $\hattensor_{K^{\circ}}$ and inside it, there is a category $K^{\circ}\Ban\Comm\Alg$ of commutative Banach $K^{\circ}$-algebras and continuous (i.e. bounded) $K^{\circ}$-algebra homomorphisms between them,
                        \item ideals of Tate $K^{\circ}$-algebras are closed with respect to the Gauss norm, which in turn implies that affinoid $K^{\circ}$-algebras are Banach $K^{\circ}$-algebras and that the category $K^{\circ}\-\Affd$ of these affinoid algebras is a full subcategory of $K^{\circ}\Ban\Comm\Alg$.
                    \end{itemize}
                
                Note also that these constructions apply more generally to all non-archimedean Banach algebras, not only $K$ and $K^{\circ}$. We leave the details up to the readers.
            \end{remark}
            \begin{definition}[Affinoid algebras topologically of finite presentation] \label{def: affinoid_algebras_topologically_of_finite_presentation}
                An affinoid $K^{\circ}$-algebra $A \cong K^{\circ}\{x_1, ..., x_n\}$ is said to be \textbf{topologically of finite presentation} (often abbreviated to \textbf{tfp}) if and only if the ideal $I$ is finitely generated\footnote{Note that being algebraically finitely generated and being topologically so are the same for ideals of affinoid algebras, since they are closed.} as a $K^{\circ}\{x_1, ..., x_n\}$-module. 
            \end{definition}
            \begin{definition}[Admissible affinoid algebras] \label{def: admissible_affinoid_algebras}
                An affinoid $K^{\circ}$-algebra that is topologically of finite presentation is said to be \textbf{admissible} over $K^{\circ}$ if and only if it is flat as a $K^{\circ}$-module. 
            \end{definition}
            The following result eseentially tells us that every affinoid $K^{\circ}$-algebra is actually already admissible in the sense of definition \ref{def: pre_admissible_and_pre_adic_rings}, thus rendering the notion of affinoid-admissibility from definition \ref{def: admissible_affinoid_algebras} (even if it were to be defined without the conditions of being topologically of finite presentation and of flatness) strictly stronger than the notion from definition \ref{def: pre_admissible_and_pre_adic_rings}. 
            \begin{proposition}[Tate algebras are topologically admissible] \label{prop: tate_algebras_over_valuation_rings_are_topologically_pre_admissible}
                Tate $K^{\circ}$-algebras are admissible in the sense of definition \ref{def: pre_admissible_and_pre_adic_rings} (i.e. as topological rings).
            \end{proposition}
                \begin{proof}
                    For this, let us firstly check whether or not Tate $K^{\circ}$-algebras are linearly topologised (cf. definition \ref{def: topological_modules}). To this end, fix once and for all a $n \in \N$ and observe that in the topology induced by the Gauss norm on the Tate $K^{\circ}$-algebra $K^{\circ}\{x_1, ..., x_n\}$, one may construct a canonical topological basis generated by the open balls $\B_{\e}(0) := \{f \in K^{\circ}\{x_1, ..., x_n\} \mid \|f\| < \e\}$ (for any $\e > 0$) which are indeed open neighbourhoods of $0$ consisting of $K^{\circ}\{x_1, ..., x_n\}$-submodules of $K^{\circ}\{x_1, ..., x_n\}$ itself: as such, $K^{\circ}\{x_1, ..., x_n\}$ is indeed linearly topologised. 
                
                    Next, let us check if $K^{\circ}\{x_1, ..., x_n\}$ has an ideal of definition (i.e. whether or not it is pre-admissible in the sense of definition \ref{def: pre_admissible_and_pre_adic_rings}); if it does, then it will automatically be admissible by virtue of being topologically complete with respect to the Gauss norm.  To this end, firstly pick an arbitrary open neighbourhood $U \ni 0$ inside $K^{\circ}\{x_1, ..., x_n\}$; we then claim that $(x_0, x_1, ..., x_n)$, with $x_0$ being the set of generators for the maximal ideal $\m_K \subset K^{\circ}$, is an ideal of definition for $K^{\circ}\{x_1, ..., x_n\}$ (it is easy to verify that this ideal is indeed open with respect to the Gauss norm). To prove this claim, we might as well firstly assume that $U := \B_{\e}(0)$ for some $\e > 0$. We can then observe that for any fixed $\e > 0$, one can always choose a positive integer $N_{\e}$ such that $\left\| \sum_{j = 0}^m f_j x_j \right\|^{N_{\e}} = \max_{0 \leq j \leq m} \|f_j\|^{N_{\e}} < r$ for all $\sum_{j = 0}^m f_j x_j \in (x_1, ..., x_n)$ (where $f_j \in K^{\circ}\{x_1, ..., x_n\}$ for every $0 \leq j \leq m$), which is because $\|f_j\| \leq 1$ for all $0 \leq j \leq m$, as their coefficients are elements of $K^{\circ}$. This shows that for every $\e > 0$, there exists a positive integer $N_{\e}$ such that any open neighbourhood $U \ni 0$ contains the ideal $(x_0, x_1, ..., x_n)^{N_{\e}}$, thus showing that $K^{\circ}\{x_1, ..., x_n\}$ is (pre-)admissible. 
                \end{proof}
            \begin{corollary}[Affinoid algebras are topologically admissible] \label{coro: affinoid_algebras_over_valuation_rings_are_topologically_pre_admissible}
                Affinoid $K^{\circ}$-algebras are admissible in the sense of definition \ref{def: pre_admissible_and_pre_adic_rings}. Furthermore, if $A := K^{\circ}\{x_1, ..., x_n\}/I$ is any affinoid $K^{\circ}$-algebra then because it is admissible in the sense of definition \ref{def: pre_admissible_and_pre_adic_rings} (and hence weakly admissible in the sense of defintion \ref{def: weakly_pre_admissible_rings}), the pair $(A, (x_1, ..., x_n)/I)$ will be a Henselian pair by proposition \ref{prop: weakly_admissible_rings_induce_henselian_pairs}.
            \end{corollary}
            \begin{convention}[Affinoid-admissibility vs. topological admissibility] \label{conv: admissibility_and_topological_admissibility}
                From now on, to distinguish between the two notions of admissibility from definitions \ref{def: admissible_affinoid_algebras} and \ref{def: pre_admissible_and_pre_adic_rings}, let us refer to the former as \say{affinoid-admissibility} or simply \say{admissibility}, whereas the latter will be known as \say{topological (pre-)admissibility}.
            \end{convention}
            \begin{remark}
                The proof of proposition \ref{prop: tate_algebras_over_valuation_rings_are_topologically_pre_admissible} tells us that unlike affinoid $K^{\circ}$-algebras, affinoid $K$-algebras are actually \textit{not} topologically (pre-)admissible as there are sufficiently large open neighbourhoods inside $K\{x_1, ..., x_n\}$ which are not contained inside any power of the ideal of definition $(x_1, ..., x_n)$. On the other hand, there is nothing preventing $K\{x_1, ..., x_n\}$ nor any other affinoid $K$-algebra from being topologically of finite presentation nor being affinoid-admissible, as these are purely ring-theoretic notions; this in essence demonstrates the necessity for the notion of affinoid-admissibility over that of topological admissibility.
            \end{remark}
            \begin{example}
                Let $\varpi \in \m_K$ be a pseudo-uniformiser and observe that due to the construction of the Gauss norm, the $\varpi$-adic completion of $K^{\circ}[x_1, ..., x_n]$ is precisely $K^{\circ}\{x_1, ..., x_n\}$, which is - of course - topologically of finite presentation and in fact, affinoid-admissible; of course, when $K^{\circ}$ is discretely valued, the maximal ideal $\m_K \subset K^{\circ}$ is \textit{a priori} principal and hence the $\varpi$-adic completion of $K^{\circ}[x_1, ..., x_n]$ is the same as its $\m_K$-adic completion, both being $K^{\circ}\{x_1, ..., x_n\}$. If $K$ is algebraically closed (or slightly weaker, if $K$ is perfectoid) then the $\m_K$-adic completion of $K^{\circ}[x_1, ..., x_n]$ will turn out to actually be $k[x_1, ..., x_n]$ of all things (!), but this can be easily proven with the observation that in such cases, $\m_K = \m_K^2$; the $K^{\circ}$-algebra $k[x_1, ..., x_n]$, however, is topologically of finite presentation if and only if $\m_K$ is finitely generated (since $k[x_1, ..., x_n] \cong K^{\circ}\{x_1, ..., x_n\}/\m_K$ as a consequence of the construction of the Gauss norm), and it is flat (and hence affinoid-admissible) if and only if $\Spec k \to \Spec K^{\circ}$ is a Zariski-open immersion (i.e. the Zariski-point $\Spec k$ inside $\Spec K^{\circ}$ is clopen). 
            \end{example}
        
        \subsubsection{Admissible formal blowups}
        
        \subsubsection{Rig-points; rigid spaces via formal schemes}