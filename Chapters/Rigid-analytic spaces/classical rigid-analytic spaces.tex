\section{Classical rigid-analytic spaces}
    \begin{convention} \label{conv: rigid_analytic_varieties_non_archimedean_ground_field}
        Henceforth, we fix a \textit{complete}\footnote{This is an important hypothesis because eventually, we will want out affinoid algebras to be instances of Banach algebra over $K$ (cf. proposition \ref{prop: affinoid_algebras_are_banach_algebras}).} non-archimedean field $(K, |-|)$ valuation ring $K^{\circ} := \{x \in K \mid |x| \leq 1\}$ and residue field $k := K^{\circ}/\m$, where $\m := \{x \in K \mid |x| < 1\}$ is the unique maximal ideal of $K^{\circ}$. In addition, we will be caring about the value group of $K$ (or rather, of $K^{\circ}$), which comes from the valuation $v: K^{\x} \to \R$ that determines the non-archimedean ultra-metric $|-| := \exp(v(-))$, and is given by $\Gamma := \im v$ (note that this is well-defined because $v: K^{\x} \to \R$ is a group homomorphism); this is a totally ordered abelian group whose ordering is inherited from the archimedean ordering of $(\R, \leq)$.
    \end{convention}

    \subsection{Tate's rigid spaces}
        \subsubsection{Affinoid algebras and the Tate topology}
            \begin{remark}
                In addition, recall that the valuation ring $K^{\circ}$ is Noetherian if and only if the value group $\Gamma$ is a discrete subgroup of $\R$; in which case, $K^{\circ}$ will be a discrete valuation ring.
            \end{remark}
            
            \begin{convention}[Multi-indices] \label{conv: mutli_indices}
                Fix a positive integer $n$. An \textbf{$n$-multi-index} is an ordered $n$-tuple $J := (j_1, ..., j_n) \in \N^n$. By $|J|$, we shall mean $\sum_{i = 1}^n j_i$ and for any ordered $n$-tuple of variables $X := (x_1, ..., x_n)$ we shall write $X^J := \prod_{i = 1}^n x_i^{j_i}$.
            \end{convention}
            \begin{definition}[Tate algebras and affinoid algebras] \label{def: affinoid_algebras}
                Let $n$ be a fixed positive integer and let $X := (x_1, ..., x_n)$ be an $n$-tuple of variables. The \textbf{$n^{th}$ Tate $K$-algebra}, denoted by $K\{X\} := K\{x_1, ..., x_n\}$, is the $K$-subalgebra\footnote{We will let the readers check for themselves that $K\{X\}$ is a $K$-subalgebra of $K[\![X]\!]$.} of $K[\![X]\!] := K[\![x_1, ..., x_n]\!]$ whose elements are the convergent\footnote{Recall that a non-archimedean power series $\sum_{|J| = 0}^{+\infty} a_J X^J$ with coefficients $a_J \in K$ converges if and only if the coefficients $a_J$ are topologically nilpotent.} power series $\sum_{|J| = 1}^{+\infty} a_J X^J$ with coefficients $a_J \in K$. This is naturally a normed algebra via the so-called \textbf{Gauss norm}, which is given by $\left\|\sum_{|J| = 0}^{+\infty} a_J X^J\right\| := \sup_J |a_J|$ and well-defined because non-archimedean convergent power series must have topologically nilpotent (hence bounded) coefficients; obviously $\|f\| = 0$ if and only if $f = 0$.
                
                An \textbf{affinoid $K$-algebra} is a quotient of a Tate $K$-algebra by an ideal thereof (i.e. Tate algebras are \say{free} affinoid algebras). 
            \end{definition}
            
            Before we move on, let us first recall several aspects of the theory of Banach spaces over non-archimedean fields, mostly to be clear about what we specfically mean by such \say{Banach spaces}.
            \begin{definition}[Non-archimedean Banach spaces] \label{def: non_archimedean_banach_spaces}
                A \textbf{non-archimedean $K$-vector space} is a $K$-vector space equipped with an ultra-norm. A \textbf{Banach space} over $K$ is a topologically complete non-archimedean $K$-vector space.
            \end{definition}
            \begin{definition}[Bounded linear maps] \label{def: bounded_linear_maps_between_non_archimedean_vector_spaces}
                A $K$-linear map $\varphi: V \to W$ between two non-archimedean $K$-vector spaces $V, W$ is said to be \textbf{bounded} if and only if there exists $C \geq 0$ such that $\|\varphi(v)\| \leq C \|v\|$ for all $v \in V$.
            \end{definition}
            \begin{remark}
                It is easy to check that the identity map on any non-archimedean $K$-vector space is bounded, and furthermore, that the boundedness of $K$-linear maps between non-archimedean $K$-vector spaces preserved under compositions.
            \end{remark}
            \begin{proposition}
                A $K$-linear map between two non-archimedean $K$-vector spaces is bounded if and only if it is continuous with respect to the topologies induced by the ultra-norms on the codomain and domain.
            \end{proposition}
                \begin{proof}
                    
                \end{proof}
            \begin{corollary}
                There is a category of (non-archimedean) $K$-Banach spaces, which we denote by $K\Ban$, whose objects are $K$-Banach spaces and whose morphisms are the continuous (i.e. bounded) $K$-linear maps.
            \end{corollary}
            \begin{definition}[Completed tensor products] \label{def: completed_tensor_products_of_non_archimedean_vector_spaces}
                The \textbf{completed tensor product} (also known as the \textbf{projective tensor product}) $V \hat\tensor_K W$ of two non-archimedean $K$-vector spaces $V$ and $W$ is the topological completion of the algebraic tensor product $V \tensor_K W$ with respect to the so-called \textbf{projective cross ultra-norm}, which for elements $x := \sum_{i = 1}^n v_i \tensor w_i \in V \tensor_K W$ is given by:
                    $$\|x\|_{\pi} := \inf\left\{ \max_{1 \leq i \leq n} \|v_i\|_V \|w_i\|_W \right\}$$
            \end{definition}
            \begin{remark}
                Of course, one ought to check that the projective cross ultra-norm as in definition \ref{def: completed_tensor_products_of_non_archimedean_vector_spaces} is indeed a well-defined ultra-norm. This, however, is easy and as such will be left to our readers as an exercise.  
            \end{remark}
            \begin{proposition}[Completed tensor products of non-archimedean Banach spaces] \label{prop: completed_tensor_products_of_non_archimedean_banach_spaces}
                The category $K\Ban$ is an infinite tensor category\footnote{This is as in \cite[Definition 4.1.1]{EGNO}, but without the local finiteness condition. In particular, this means that $K\Ban$ is symmetric monoidal, rigid, and closed with respect to the completed tensor product of non-archimedean $K$-vector spaces.} whose monoidal structure is the completed tensor product $\hattensor_K$ of non-archimedean $K$-vector spaces.
            \end{proposition}
                \begin{proof}
                    
                \end{proof}
            \begin{corollary}
                There exists a category of commutative Banach $K$-algebras internal to $K\Ban$, whose objects are the commutative monoids internal to $K\Ban$ and whose morphisms are the continuous (i.e. bounded) $K$-algebra homomorphisms between them. This category shall be denoted by $K\Ban\Comm\Alg$. 
            \end{corollary}
            
            \begin{lemma}[Quotients of non-archimedean Banach spaces] \label{lemma: quotients_of_non_archimedean_banach_spaces}
                Let $V$ be a $K$-Banach space and $V_0 \subseteq V$ be a closed $K$-vector subspace thereof. Then $V/V_0$ will also be a $K$-Banach space. 
            \end{lemma}
                \begin{proof}
                    
                \end{proof}
            \begin{lemma}[Ideals of Tate algebras are closed] \label{lemma: tate_algebra_ideals_are_closed}
                Ideals inside Tate algebras are closed with respect to the topology induced by the Gauss norm.
            \end{lemma}
                \begin{proof}
                    
                \end{proof}
            \begin{proposition}[Affinoid algebras are Banach algebras] \label{prop: affinoid_algebras_are_banach_algebras}
                Affinoid algebras are Banach $K$-algebras when equipped with the Gauss norm (note that this relies crucially on the assumption that $K$ is complete as a non-archimedean field).
            \end{proposition}
                \begin{proof}
                    Thanks to lemma \ref{lemma: quotients_of_non_archimedean_banach_spaces} and \ref{lemma: tate_algebra_ideals_are_closed}, it suffices to only prove that any given Tate algebra $K\{x_1, ..., x_n\}$ (for any positive integer $n$) is a Banach $K$-algebra with respect to the topology induced by the Gauss norm; in fact, because $K\{x_1, ..., x_n\} \cong K\{x_1\} \hattensor_K ... \hattensor_K K\{x_n\}$. For this, pick a Cauchy sequence $\{f_i\}_{i \in \N}$ of single-variable convergent power series $f_i(x) := \sum_{d = 0}^{+\infty} a_d^{(i)} x^d \in K\{x\}$ and observe that because:
                        $$\|f_i(x) - f_j(x)\| = \left\| \sum_{d = 0}^{+\infty} (a_d^{(i)} - a_d^{(j)}) x^d \right\| = \sup_d \|a_d^{(i)} - a_d^{(i)}\|$$
                    the sequence of coefficients $\{a_d^{(i)}\}_{i \in \N}$ is also Cauchy at each degree $d \geq 0$ as a result of the assumption that $\{f_i\}_{i \in \N}$ is Cauchy. Because $K$ is complete by hypothesis, the sequences of coefficients $\{a_d^{(i)}\}_{i \in \N}$ all converge by virtue of being Cauchy. Now, observe that:
                        $$\underset{i \to +\infty}{\lim} f_i(x) = \underset{i \to +\infty}{\lim} \sum_{d = 0}^{+\infty} a_d^{(i)} x^d = \sum_{d = 0}^{+\infty} \left( \underset{i \to +\infty}{\lim} a_d^{(i)} \right) x^d$$
                    per some elementary non-archimedean analysis. As such, the sequence $\{f_i\}_{i \in \N}$ is convergent, and since it is an arbitrary Cauchy sequence, means that $K\{x\}$ (and by extension, any Tate $K$-algebra $K\{x_1, ..., x_n\}$) is complete with respect to the topology induced by the Gauss norm. 
                \end{proof}
            \begin{corollary}
                There is a full subcategory of the category $K\Ban\Comm\Alg$ of commutative Banach $K$-algebras and continuous $K$-algebra homomorphisms, wherein the objects are affinoid $K$-algebras. This category shall be denoted by $K\-\Affd$.
            \end{corollary}
            \begin{proposition}[Maximal spectra of affinoid algebras] \label{prop: maximal_spectra_of_affinoid_algebras}
                There is a functor: $\Spm: K\-\Affd \to \Sets$ which associates to each affinoid $K$-algebra $A$ the set $\Spm A$ whose elements are the maximal ideals of $A$. 
            \end{proposition}
                \begin{proof}
                    
                \end{proof}
        
        \subsubsection{Structure sheaves of Tate's rigid spaces}
        
        \subsubsection{Coherent sheaves on Tate's rigid spaces}
    
    \subsection{Raynaud's formal models for rigid spaces}
        \subsubsection{Coherent sheaves on formal schemes topologically of finite presentations}
            \begin{remark}[Affinoids over complete valuation rings] \label{remark: affinoids_over_complete_valuation_rings}
                Since $(K, |-|)$ (as in convention \ref{conv: rigid_analytic_varieties_non_archimedean_ground_field}) is topologically complete with respect to the topology induced by the ultra-metric $|-|$, and since $K^{\circ}$ is a closed topological subspace, it is also topologically complete with respect to $|-|$. As such, many functional-analytic results concerning affinoid $K$-algebras carry over \textit{verbatim} to so-called \textbf{affinoid $K^{\circ}$-algebras}: these are quotients of \textbf{Tate $K^{\circ}$-algebras}, i.e. they are of the form $K^{\circ}\{x_1, ..., x_n\}/I$ for some positive integer $n$ and some $K^{\circ}\{x_1, ..., x_n\}$-ideal $I$.
                
                In particular, this means that:
                    \begin{itemize}
                        \item there is an infinite tensor category $K^{\circ}\Ban$ of Banach (non-archimedean) $K^{\circ}$-modules and continuous (i.e. bounded\footnote{In fact, bounded by $1$.}) $K^{\circ}$-linear maps where the monoidal structure is given by $\hattensor_{K^{\circ}}$ and inside it, there is a category $K^{\circ}\Ban\Comm\Alg$ of commutative Banach $K^{\circ}$-algebras and continuous (i.e. bounded) $K^{\circ}$-algebra homomorphisms between them,
                        \item ideals of Tate $K^{\circ}$-algebras are closed with respect to the Gauss norm, which in turn implies that affinoid $K^{\circ}$-algebras are Banach $K^{\circ}$-algebras and that the category $K^{\circ}\-\Affd$ of these affinoid algebras is a full subcategory of $K^{\circ}\Ban\Comm\Alg$.
                    \end{itemize}
            \end{remark}
            \begin{definition}[Affinoid algebras topologically of finite presentation] \label{def: affinoid_algebras_topologically_of_finite_presentation}
                An affinoid $K^{\circ}$-algebra $A \cong K^{\circ}\{x_1, ..., x_n\}$ is said to be \textbf{topologically of finite presentation} (often abbreviated to \textbf{tfp}) if and only if the ideal $I$ is finitely generated\footnote{Note that being algebraically finitely generated and being topologically so are the same for ideals of affinoid algebras, since they are closed.} as a $K^{\circ}\{x_1, ..., x_n\}$-module. 
            \end{definition}
            \begin{definition}[Admissible affinoid algebras] \label{def: admissible_affinoid_algebras}
                An affinoid $K^{\circ}$-algebra that is topologically of finite type is said to be \textbf{admissible} over $K^{\circ}$ if and only if it is flat as a $K^{\circ}$-module. 
            \end{definition}
            \begin{remark}
                Let us compare definitions \ref{def: admissible_affinoid_algebras} and \ref{def: pre_admissible_and_pre_adic_rings}: specifically, we would like to know if the notion of admissibility for affinoid algebras coincides with (or at least, is closedly related to) that of linearly topologised rings.
                
                For this, let us firstly check whether or not affinoid $K^{\circ}$-algebras are linearly topologised, because otherwise it would not make much sense to check whether or not it is admissible in the sense of definition \ref{def: pre_admissible_and_pre_adic_rings}. To this end, observe that in the topology induced by the Gauss norm on the Tate $K^{\circ}$-algebra $K^{\circ}\{x_1, ..., x_n\}$, one may construct a canonical topological basis generated by the open balls $B_r(0) := \{f \in K^{\circ}\{x_1, ..., x_n\} \mid \|f\| < r\}$ (for any $r \in \R_{> 0}$) which are indeed open neighbourhoods of $0$ consisting of $K^{\circ}\{x_1, ..., x_n\}$-submodules of $K^{\circ}\{x_1, ..., x_n\}$ itself: as such, $K^{\circ}\{x_1, ..., x_n\}$ is indeed linearly topologised. 
                
                Next, let us check if $K^{\circ}\{x_1, ..., x_n\}$ has an ideal of definition (i.e. whether or not it is pre-admissible in the sense of definition \ref{def: pre_admissible_and_pre_adic_rings}); if it does, then it will automatically be admissible by virtue of being topologically complete with respect to the Gauss norm.  To this end, firstly pick an arbitrary open neighbourhood $U \ni 0$ inside $K^{\circ}\{x_1, ..., x_n\}$; we then claim that $(x_1, ..., x_n)$ is an ideal of definition for $K^{\circ}\{x_1, ..., x_n\}$ (it is easy to verify that this ideal is indeed open with respect to the Gauss norm). To prove this claim, we might as well assume that $U := B_r(0)$ for some $r \in \R_{> 0}$; then we make the following observation:
                    $$$$
            \end{remark}
        
        \subsubsection{Admissible formal blowups}
        
        \subsubsection{Rig-points; rigid spaces via formal schemes}